\documentclass[12pt]{article}
\usepackage[backend=bibtex,style=numeric]{biblatex} 
\addbibresource{references.bib} 

\title{Report for the CPS Final Project}
\author{
    Andrea Auletta \\ \texttt{andrea.auletta@studenti.unipd.it} \and
    Niccolò Zenaro \\ \texttt{niccolo.zenaro@studenti.unipd.it}
}
\begin{document}

\maketitle
\newpage
\tableofcontents
\newpage

\begin{abstract}
Modbus is a commonly used protocol in Supervisory Control And Data Acquisition (\textit{SCADA}) environments
for monitoring, control and data acquisition. Despite its wide popularity, Modbus is not secure because when it was developed and adopted ($1979$) security was not considered to be a concern in isolated Industrial Control Systems (\textit{ICS}), thus is not designed to be secure like modern IT networks.
Among the various attacks, 4 different taxonomies can be identified to facilitate formal risk analysis efforts for clarifying the nature and the scope of the security threats on Modbus systems and networks.
\end{abstract}

\section{Introduction and objectives}
The Modicon Commuication Bus (\textit{Modbus}) protocolo operates in a master-slave or server-client based model. The master devices initiates the queries while the slave devices respond to all such queries.
Masters can either send a broadcast message to all the slaves or indivdually poll a specific device. 
All the experiments run in this work, like in the original paper \cite{huitsing2008attack} are focused on TCP/IP implementation, while Modbus protocol can be implemented also on top of several communication networks like serial or UDP. 

\section{System setup}
\section{Experiments}
\section{Results and Discussion}
\printbibliography 


\end{document}
